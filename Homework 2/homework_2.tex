\documentclass{article}
\usepackage[hscale=0.7,vscale=0.8]{geometry}
\usepackage[utf8]{inputenc}
\usepackage{listings}
\usepackage{xcolor}
\usepackage{enumitem,amssymb}
\usepackage[T1]{fontenc}
\usepackage{mathpazo}  % Palatino font

\usepackage{microtype} % Improves spacing & kerning
\usepackage{hyperref}

% Define custom colors
\definecolor{codebg}{RGB}{240,240,240}  % Light gray background
\definecolor{codeborder}{RGB}{100,100,100}  % Darker gray border
\definecolor{keywordcolor}{RGB}{0,0,150}  % Dark blue for keywords
\definecolor{commentcolor}{RGB}{0,128,0}  % Green for comments
\definecolor{stringcolor}{RGB}{150,0,0}  % Dark red for strings
\hypersetup{
    colorlinks=true,        % Enable colored links
    urlcolor=magenta,          % Change link color
    linkcolor=red,          % Color for internal links
    citecolor=green         % Color for citations
}

% Define custom lstlisting style
\lstdefinestyle{customcode}{
    backgroundcolor=\color{codebg},
    frame=single, 
    rulecolor=\color{codeborder},
    keywordstyle=\color{keywordcolor}\bfseries,
    commentstyle=\color{commentcolor},
    stringstyle=\color{stringcolor},
    basicstyle=\ttfamily\footnotesize,
    breaklines=true,
    framextopmargin=5pt,
    framexbottommargin=5pt,
    xleftmargin=10pt,
    xrightmargin=10pt,
    showstringspaces=false,
    numbers=left,
    numberstyle=\tiny\color{gray},
}
\title{Homework 2: Introduction to Django}
\author{}
\date{}

\begin{document}
\maketitle

\section{Objective}
You are tasked with building a RESTful Movie Theater Booking Application using Python and Django. The application should allow users to:

\begin{itemize}
    \item View movie listings
    \item Book seats
    \item Check their booking history through a RESTful API
\end{itemize}

Additionally, you will create an attractive user interface using Django templates and Bootstrap.

\section{Time Requirement}
Approximately 4-5 hours.

\section{Instructions}
\subsection{Project Setup (30 minutes)}
\begin{enumerate}
    \item Create a new Django project named \texttt{movie\_theater\_booking}.
    \item Set up a virtual environment and install dependencies:
    \begin{lstlisting}[style=customcode,language=bash]
    python3 -m venv myenv
    source myenv/bin/activate
    pip install django djangorestframework
    \end{lstlisting}
\end{enumerate}

\subsection{Creating the Booking App (30 minutes)}
Inside your project, create a Django app named \texttt{bookings} with the following models:

\begin{itemize}
    \item \textbf{Movie}: title, description, release date, duration.
    \item \textbf{Seat}: seat number, booking status.
    \item \textbf{Booking}: movie, seat, user, booking date.
\end{itemize}

\subsection{Implementing the MVT Architecture (1 hour)}
\begin{itemize}
    \item Create and run migrations.
    \item Develop RESTful views using Django REST Framework.
    \item Set up templates for movie listings, seat booking, and booking history.
\end{itemize}

\subsection{Creating an Attractive User Interface (1 hour)}
Use Bootstrap to create responsive templates:
\begin{lstlisting}[style=customcode,language=html]

Movie List

<h2>Available Movies</h2>
<ul class="list-group">
    
        <li class="list-group-item">
            <h5>{{ movie.title }}</h5>
            <p>{{ movie.description }}</p>
            <a href="" class="btn btn-primary">Book Now</a>
        </li>
    
</ul>

\end{lstlisting}

\subsection{RESTful API Implementation (1 hour)}
Create serializers and define API endpoints:
\begin{itemize}
    \item \texttt{/api/movies/} : List and manage movies.
    \item \texttt{/api/seats/} : Check and book seats.
    \item \texttt{/api/bookings/} : View and create bookings.
\end{itemize}

\subsection{Testing (30 minutes)}
Write unit tests for models and test API responses.

\subsection{Deployment (30 minutes)}
Deploy using Django's development server:
\begin{lstlisting}[style=customcode,language=bash]
python manage.py runserver 0.0.0.0:3000
\end{lstlisting}

\section{Submission Requirements}
Submit a zip file containing:
\begin{itemize}
    \item Project source code with tests.
    \item README file with setup instructions.
    \item GitHub repository with frequent commits.
\end{itemize}

\section{Assessment Criteria}
\begin{itemize}
    \item \textbf{Functionality (35 points)}: Meets all requirements.
    \item \textbf{User Experience (15 points)}: Clean UI with Bootstrap.
    \item \textbf{Code Quality (15 points)}: Follows Django conventions.
    \item \textbf{Testing (20 points)}: Unit and integration tests included.
    \item \textbf{Deployment (10 points)}: Accessible via DevEdu.
    \item \textbf{Documentation (5 points)}: Clear README included.
\end{itemize}

\end{document}
