
\documentclass{article}
\usepackage[hscale=0.7,vscale=0.8]{geometry}
\usepackage[utf8]{inputenc}
\usepackage{listings}
\usepackage{xcolor}
\usepackage{enumitem,amssymb}
\usepackage[T1]{fontenc}
\usepackage{mathpazo}  % Palatino font
\usepackage{microtype} % Improves spacing & kerning
\usepackage{hyperref}

% Define custom colors
\definecolor{codebg}{RGB}{240,240,240}  % Light gray background
\definecolor{codeborder}{RGB}{100,100,100}  % Darker gray border
\definecolor{keywordcolor}{RGB}{0,0,150}  % Dark blue for keywords
\definecolor{commentcolor}{RGB}{0,128,0}  % Green for comments
\definecolor{stringcolor}{RGB}{150,0,0}  % Dark red for strings
\hypersetup{
    colorlinks=true,        % Enable colored links
    urlcolor=magenta,          % Change link color
    linkcolor=red,          % Color for internal links
    citecolor=green         % Color for citations
}

% Define custom lstlisting style
\lstdefinestyle{customcode}{
    backgroundcolor=\color{codebg},
    frame=single, 
    rulecolor=\color{codeborder},
    basicstyle=\ttfamily\small,
    keywordstyle=\color{keywordcolor}\bfseries,
    commentstyle=\color{commentcolor}\itshape,
    stringstyle=\color{stringcolor},
    breaklines=true,
    numbers=left,
    numberstyle=\tiny\color{gray},
    captionpos=b,
    tabsize=4,
    showstringspaces=false
}

\lstset{
    basicstyle=\ttfamily\color{blue}, % Custom styling
}

\newlist{todolist}{itemize}{2}
\setlist[todolist]{label=$\square$}

\title{Sprint 4: Refactoring and Final Touches}
\author{}
\date{}

\usepackage{natbib}
\usepackage{graphicx}

\begin{document}
\maketitle
\section*{Sprint 4 Objectives}

\begin{enumerate}[label=\arabic*)]
\item Project in GitHub (0~pts)
\item Features with tests (10~pts)
\item 80\% Test coverage metrics (15~pts)
\item CI Pipeline
    \begin{enumerate}
        \item AI Code Review using OpenAI Platform for each Pull Request (2.5~pts)
        \item Running automated tests for each commit (2.5~pts)
        \item Reports test coverage metrics in console for each commit (2.5~pts)
        \item PyLint and/or Flake~8: Linting scans in CI pipeline with reports (10~pts)
        \item Dependabot: Dependency vulnerability scans in CI pipeline with reports (10~pts)
    \end{enumerate}
\item Deployed to a production environment
    \begin{enumerate}
        \item Custom domain name (no subdomains unless you own the top‑level domain) (5~pts)
    \end{enumerate}
\item CD Pipeline that runs when code is merged into \texttt{master}/\texttt{main}
    \begin{enumerate}
        \item Continually deployed to production environment using automated actions (2.5~pts)
    \end{enumerate}
\item \textbf{NEW:} 100\% of vulnerabilities and linter findings are addressed and fixed (20~pts)
\item \textbf{NEW:} 5‑minute marketing video uploaded to YouTube showcasing the application (20~pts)
    \begin{enumerate}
        \item Loom offers free screen‑recording options
    \end{enumerate}
\end{enumerate}

\section*{Demo Expectations}

\begin{enumerate}[label=\arabic*)]
\item Share the URL to your production deployment so we can follow along.
\item Tell us about your application and why you’re making it.
\item Go through Zenhub / GitHub Projects and show us what work you completed based on user story.
\item Demo the work you completed on the production environment showing the feature in production and the CD pipeline and test cases used to test each requirement.
\item Get customer feedback.
\end{enumerate}

\section*{Sprint Duration}

Each sprint is two weeks long. Make sure that you work through both weeks of the sprint to achieve your tasks!

\section*{Common Sprint Requirements}

\subsection*{Backlog Grooming and Prioritization}
At the beginning of the sprint, meet with your group to groom the backlog and prioritize work. Mark stories that are MVPs (minimum viable product).

\subsection*{Planning and Story Division}
Begin the sprint by reviewing your backlog. Add or divide stories as needed—this may include playing Planning Poker to determine story points. Record all changes in Zenhub / GitHub. Meet with your customer to identify the stories they want you to implement. Prioritize and confirm with them. Select as many stories as you hope to complete in the sprint. Balance workload between sprints; each person should have at least one story. At least two larger stories should be implemented and fully tested each sprint. Move the selected tasks from \emph{Backlog} to \emph{Sprint 1} (or the current sprint). Each group member should take ownership of \textbf{at least} one task. Keep the \emph{FIRST} and \emph{SMART} principles in mind. MVPs should be prioritized.

\subsection*{Testing and Development Process}
All stories selected for a sprint should be assigned to a main person and include unit/integration tests. Follow the \emph{Red‑Red‑Green‑Green‑Refactor} process (refactoring will be evaluated at the end of the semester). You are responsible for full‑stack development (front end and back end) and associated integration and unit testing. 
Use branches and communicate clearly. Push, pull, and merge to \texttt{main} frequently. Branch protections are in place; approval is required for merges. \textbf{Work early} so your branch is submitted in time. As you finish tasks, move them along to completion in your project‑management tool.

\section*{Submission}

\subsection*{On Canvas (one team member)}
\begin{itemize}
\item Submit your GitHub repository URL.
\item Submit your application's production URL.
\item Submit a short report of what was implemented and tested, noting any incomplete items and plans.
\item Explain any changes to your plan since the previous submission.
\item Document issues with tests, if any.
\item Record your velocity (as reported in Zenhub) at the end of the document.
\end{itemize}

\subsection*{On GitHub}
\begin{itemize}
\item Ensure all project files, tests, and code are committed and pushed to the \texttt{main} branch.
\item All students must push to GitHub with correctly configured usernames.
\item Tag the repository revision with \verb|sprint<i>| where \verb|<i>| is the sprint number.
\begin{verbatim}
git tag -a sprint<i>
git push --tags
\end{verbatim}
\item Verify that tags are available on GitHub before submission.
\end{itemize}

\subsection*{On Zenhub/GitHub Projects}
Show your point assignments, assignees, and progress as described.

\subsection*{Discord}
One team member must post to your \texttt{Dev\_Cust} group detailing what was implemented and your app URL. Customers can respond there.

\subsection*{AI Documentation}
Document in your project \texttt{README} where and when you received AI assistance, including transcript URLs for each instance.

\subsection*{Kritik Evaluations}
Complete your Kritik evaluations. The first opens immediately after the sprint deadline; each activity has three stages. Check the schedule for deadlines.

\end{document}
